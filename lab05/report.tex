\include{settings}

\begin{document}	% начало документа

% Титульная страница
\include{titlepage}

% Содержание
\include{ToC}
\section{Цель работы}
Изучение частотной и фазовой модуляции и демодуляции сигналов.
\section{Постановка задачи}
\begin{enumerate}
\item  сгенерировать однотональный сигнал низкой частоты 
\item  выполнить фазовую модуляцию и демодуляцию 
\item  выполнить частотную модуляцию и демодуляцию 
\item  получить спектр модулированного сигнала
\end{enumerate}

\section{Теоретическая информация}

\subsection{Модуляция}
Модуляция - это перенос спектра сигналов из низкочастотной области на заданную частоту. 
Это применяется для передачи сигнала в заданном частотном диапазоне.
Для модулирующего (исходного) сигнала $ S(t) $ в канале связи для передачи формируется  вспомогательный периодический высокочастотный сигнал $u(t)=f(t, [a_1,   a_2,   ...   a_m])$. Параметры $a_i$ определяют форму сигнала. 
При модуляции исходный сигнал $S(t)$ переносят на один из параметров $a_i$, форма сигнала $u(t)$ (несущей) изменяется и 
служит для переноса информации, содержащейся в сигнале $S(t)$. Обратная операция выделения сигнала $S(t)$ из 
модулированного сигнала $u(t)$ называется демодуляция.

\subsection{Однотональный сигнал}

Для генерации гармонического сигнала можно воспользоваться формулой\\ $signal = A*cos(2*\pi * f*t + \varphi)$,
 где $ A $ --- амплитуда сигнала, $f$ --- частота, $t$ --- вектор отсчетов времени, $\varphi$ --- смещение по фазе.

\subsection{Угловая модуляция}

При угловой модуляции в несущем гармоническом колебании $u(t) = U_m cos(\omega t + \varphi)$ 
значение амплитуды колебаний $U_m$ остается постоянным, а информация $s(t)$ переносится либо на частоту $\omega$, 
либо на фазовый угол $\varphi$. В обоих случаях текущее значение фазового угла гармонического 
колебания $u(t)$ определяет аргумент $\psi (t) = \omega t + \varphi$ ,
который называется полной фазой колебания.

\subsubsection{Фазовая модуляция}
При фазовой модуляции модулирующий сигнал определяет фазу несущего колебания
$\phi(t) = k s(t)$. Сигнал с фазовой модуляцией имеет вид 
\begin{equation}
    u(t) = U_m \cos(\omega_0 t + k s(t))
\end{equation}


\subsubsection{Частотная модуляция}
При частотной модуляции модулирующий сигнал определяет частоту несущего колебания.
Сигнал с частотной модуляцией имеет вид  
\begin{equation}
	u(t) = U_m cos(\omega_0 t + k \int_{0}^{t} s(t) dt)
\end{equation}

\section{Ход выполнения работы}

\subsection{Однотональный сигнал}
Получим обычный гармонический сигнал  $s(t) = Acos(2 \pi ft + \phi)$ 
\begin{figure}[H]
	\begin{center}
		\includegraphics[scale=0.7]{003_origin_sig.png}
		\caption{Однотональный сигнал} 
		\label{pic:001} % название для ссылок внутри кода
	\end{center}
\end{figure}
Построим его спектр:
\begin{figure}[H]
	\begin{center}
		\includegraphics[scale=0.7]{001_origin_spec.png}
		\caption{Однотональный сигнал} 
		\label{pic:002} % название для ссылок внутри кода
	\end{center}
\end{figure}

\subsection{Фазовая модуляция}
Промодулируем и также получим спектр:
\begin{figure}[H]
	\begin{center}
		\includegraphics[scale=0.7]{002_mod_spec.png}
		\caption{Спектр модулированного сигнала} 
		\label{pic:003} % название для ссылок внутри кода
	\end{center}
\end{figure}
Проведем демодуляцию и посмотрим результаты:
\begin{figure}[H]
	\begin{center}
		\includegraphics[scale=0.7]{004_recovered.png}
		\caption{Демодулированный сигнал(фазовая модуляция)} 
		\label{pic:004} % название для ссылок внутри кода
	\end{center}
\end{figure}
Cпектр:
\begin{figure}[H]
	\begin{center}
		\includegraphics[scale=0.7]{005_demod_spec.png}
		\caption{Спектр демодулированного сигнала(фазовая демодуляция)} 
		\label{pic:005} % название для ссылок внутри кода
	\end{center}
\end{figure} 
Очевидно, что исходный сигнал совпадает с демодулированным, получившийся спектр также это подтверждает.   

\subsection{Частотная модуляция}
Промодулировав частотно наш сигнал, получили следующий график:
\begin{figure}[H]
	\begin{center}
		\includegraphics[scale=0.7]{006_mod_sig.png}
		\caption{Модулированного сигнала(частотная модуляция)} 
		\label{pic:006} % название для ссылок внутри кода
	\end{center}
\end{figure} 
Далее получили демодулированный сигнал, для которого вычислили преобразование Фурье и постороили амплитудный спектр.
\begin{figure}[H]
	\begin{center}
		\includegraphics[scale=0.7]{007_demod_spec.png}
		\caption{Спектр демодулированного сигнала(частотная модуляция)} 
		\label{pic:007} % название для ссылок внутри кода
	\end{center}
\end{figure} 

Как и в случае с фазовой модуляцией-демодуляцией исходный сигнал демодулирован хорошо.
Код всей лабораторной работы на Matlab приведен ниже.

\lstinputlisting[
	label=code:lab05,
	caption={lab05.m},% для печати символ '_' требует выходной символ '\'
]{lab05.m}
\parindent=1cm % командна \lstinputlisting сбивает параментры отступа

\section{Выводы}
Таким образом нам удалось на качественном уровне понять основы угловой модуляции и выполнить модулирование совместно с демодулированием однотоннального сигнала. Экспериментально удалось подтвердить, что данные типы модуляции на хорошем уровне позволяют передавать сигнал. Также следует отметить, что фазовая модуляция совместно с частотной модуляцией в связи с высоким КПД и другими свойствами достаточно часто используются в Радиолокации. А значит полученные нами знания в результате выполнения данной работы пригодятся в самом обозримом будующем - на военной кафедре:).   
\end{document}
