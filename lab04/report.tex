\include{settings}

\begin{document}	% начало документа

% Титульная страница
\include{titlepage}

% Содержание
\include{ToC}

\section{Цель работы}
Изучить амплитудную модуляцию/демодуляцию сигнала.

\section{Постановка задачи}
\begin{itemize}
	\item Сгенерировать однотональный сигнал низкой частоты.
	\item Выполнить амплитудную модуляцию.
	\item Получить спектр модулированного сигнала.
	\item Выполнить модуляцию с подавлением несущей.
	\item Выполнить однополосную модуляцию.
	\item Выполнить синхронное детектирование и получить исходный однополосный сигнал.
	\item Рассчиать КПД.
\end{itemize}


\section{Теоретическая информация}
\subsection{Модуляция}
Модуляция(\href{https://ru.wikipedia.org/wiki/Модуляция}{wiki}) — процесс изменения одного или нескольких параметров высокочастотного несущего колебания по закону низкочастотного информационного сигнала (сообщения). Сущность модуляции заключается в следующем. Формируется некоторое колебание (чаще всего гармоническое), называемое несущим колебанием или просто несущей (carrier), и какой-либо из параметров этого колебания изменяется во времени пропорционально исходному сигналу. Исходный сигнал называют модулирующим (modulating signal), а результирующее колебание с изменяющимися во времени параметрами — модулированным сигналом (modulated signal). Обратный процесс — выделение модулирующего сигнала из модулированного колебания — называется демодуляцией (demodulation).
Обычнно за несущий сигнал принимается гармоническое колебание вида 
$ s(t) = A \cos(\omega_0 t + \phi_0) $, где $ А $ - амплитуда, $ \omega_0 $ - циклическая частота, $t$ - время, $ \phi_0 $ - начальная фаза. Какие параметры можно использовать для процесса модуляции - 
очевидно те, которые не являются изменяемыми аргументами(параметрами) исходного несущего колебания - амплитуда, частота, фаза. Ну и по названию существуют модуляции:

\begin{itemize}
	\item амплитудная
	\item частотная
	\item фазовая
\end{itemize}   

Частотную и фазовую модуляции часто называют угловым видом модуляции, т.к. они меняют аргумент косинуса. Так же существует так называемая квадратурная модуляция, при которой изменяются, и амплитуда, и фаза сигнала. 
В данной работе мы рассматриваем только амплитудную модуляцию.

\subsection{Амплитудная модуляция}
Как явствует из названия, при амплитудной модуляции (AM; английский термин — amplitude modulation, AM) в соответствии с модулирующим сигналом изменяется амплитуда несущего колебания:
$ u_{am}(t) =  A(t) \cos( \omega_0 t + \phi_0)$.
Однако если амплитуду A(t) просто сделать прямо пропорциональной модулирующему сигналу, возможно возникновение следующей проблемы. Как правило, модулирующий сигнал является двуполярным (знакопеременным). Экспериментально доказано, что амплитудная огибающая, которая будет выделена в процессе демодуляции, в данном случае оказывается неправильной и соответсвует модулю исходного сигнала. Поэтому при реализации АМ к модулирующему сигналу предварительно добавляют постоянную составляющую, чтобы сделать его однополярным: $ A(t) =  A_0 + ks_m(t) $.
В итоге формула амплитудной модуляции в общем виде выглядит так: 
\begin{equation}\label{eq01}
	u(t) = (1 + M U_m cos(\Omega t)) cos(\omega_0 t + \phi _0)
\end{equation}
Спектр сигнала с амплитудной модуляцией показан на рис.\ref{pic:pic001_spec_amtheor}.
На графике $\omega_0$ --- частота несущей, $\Omega$ --- частота модуляции.
\begin{figure}[H]
	\begin{center}
		\includegraphics[scale=0.7]{spec_a_mod.png}
		\caption{Спектр амплитудно-модулированного сигнала} 
		\label{pic:pic001_spec_amtheor} % название для ссылок внутри кода
	\end{center}
\end{figure}

Амплитудная модуляция имеет низкий КПД.

\subsubsection{Амплитудная модуляция с подавлением несущей}
Основная мощность АМ сигнала приходится на несущую частоту. При АМ с подавлением несущей производится перемножение двух сигналов – модулирующего и несущего. В результате несущая частота подавляется и КПД модуляции становится 100\%.
Формула такой модуляции:
\begin{equation}\label{eq02}
	u(t) = M U_m cos(\Omega t) cos(\omega_0 t + \phi _0)
\end{equation}

Спектр сигнала с амплитудной модуляцией с подавлением несущей отличается от спектра модуляции без подавления несущей только наличием основной (несущей гармоники), которая и определяет потери в энергии. 

\subsubsection{Однополосная модуляция}
Рассмотренная в предыдущем разделе двухполосная AM с подавленной несущей имеет преимущества перед обычной AM только в энергетическом смысле — за счет устранения несущего колебания. Ширина же спектра при этом остается равной удвоенной частоте модулирующего сигнала.
Однако можно легко заметить, что спектры двух боковых полос АМ-сигнала являются зеркальным отражением друг друга, то есть они несут одну и ту же информацию. Поэтому одну из боковых полос можно удалить. Получающаяся модуляция называется однополосной (английский термин — single side band, SSB).
В зависимости от того, какая боковая полоса сохраняется, говорят ободнополосной модуляции с использованием верхней или нижней боковой полосы. Формирование однополосного сигнала проще всего пояснить, приведя несколько спектральных графиков:  

\begin{figure}[H]
	\begin{center}
		\includegraphics[scale=0.7]{spec_one_stripe.png}
		\caption{Однополосная модуляция: а - спектр модулирующего сигнала,
		б - спектр однополсоного сигнала с верхней боковой полосой, в - то же с нижней боковой 			        полосой} 
		\label{pic:spec_one_stripe} % название для ссылок внутри кода
	\end{center}
\end{figure}

По сути дела, при однополосной модуляции происходит просто сдвиг спектра сигнала в окрестности частоты несущего колебания. В отличие от AM каждая «половинка» спектра смещается в своем направлении: область положительных частот — к $ \omega_0 $, а область отрицательных частот — к $  - \omega_0 $
Очевидно, что ширина спектра однополоcного сигнала равна ширине спектра модулирующего сигнала. Таким образом, спектр однополосного сигнала оказывается в два раза уже, чем при обычной АМ.

\subsubsection{Демодуляция с помощью синхронного детектирования}
При синхронном детектировании модулированный сигнал умножается на опорное колебание с частотой несущего колебания:
 \begin{equation}\label{eq04}
 	y(t) = U(t) cos(\omega_0 t) cos(\omega_0 t) = \frac{U(t)}{2} (1 + cos(2\omega_0 t))
\end{equation}
Сигнал разделяется на два слагаемых, первое из которых повторяет исходный модулирующий сигнал, а второе повторяет модулированный сигнал на удвоенной несущей частоте 2$\omega_0$. 

Амплитудный спектр сигналов после демодуляции однозначно соотносится со спектром входного модулированного сигнала: 
амплитуды гармоник модулированного сигнала на частоте 2$\omega_0$ в два раза меньше амплитуд входного сигнала,
 постоянная составляющая равна амплитуде несущей частоты $\omega_0$ и не зависит от глубины модуляции, амплитуда
  информационного демодулированного сигнала в два раза меньше амплитуды исходного модулирующего сигнала. 

Особенностью синхронного детектирования является независимость от глубины модуляции,
 т.е. коэффициент модуляции сигнала может быть больше единицы. 
 При синхронном детектировании требуется точное совпадение фаз и частот 
 опорного колебания демодулятора и несущей гармоники АМ сигнала.

\subsubsection{КПД модуляции}
КПД амплитудной модуляции зависит от коэффициента модуляции и может быть рассчитано по следующей формуле:
 \begin{equation}\label{eq05}
	\eta (t) =\frac{ U_m^2(t) M^2}{4 P_U}  = \frac{M^2}{2 + M^2} 
\end{equation}


\section{Ход работы}




\section{Выводы}
\LaTeX\ удобен для создания отчётов, так как сам следит за нумерацией таблиц, рисунков, листингов и отсылок к ним (так, например, здесь всегда будет указан номер рисунка "sample" не зависимо от того, какой он (1,2 или другой) - это рисунок \ref{pic:pic_name}). Не менее важно что весь документ оформлен в едином стиле, а исходные материалы подключаются к отчёту, а не хранятся в нём. Всё это позволяет легко получить качественный отчёт без дополнительных трат на его офрмление.

Исключения, пожалуй, составляют таблицы, так как их значительно сложнее создавать кодом, нежели в графическом редакторе. Но здесь никто не запрещает использовать визуальные средства создания таблиц для \LaTeX\ .
\end{document}
