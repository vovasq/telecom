\include{settings}

\begin{document}    % начало документа


% Титульная страница
\include{titlepage}

% Содержание
\include{ToC}



\section{Цель работы}
Изучение методов помехоустойчивого кодирования и сравнения их свойств.
 
\section{Постановка задачи}
Провести кодирование/декодирование сигнала, полученного с помощью функции randerr кодом Хэмминга 2-мя способами: с помощью встроенных функций encode/decode, а также через создание проверочной и генераторной матриц и вычисление синдрома. Оценить корректирующую способность кода.

Выполнить кодирование/декодирование циклическим кодом, кодом БЧХ, кодом Рида-Соломона. Оценить корректирующую способность кода.

 
\section{Теоретическая информация}

\subsection{Кодирование}
Физическое кодирование — линейное преобразование двоичных данных, осуществляемое для их передачи по физическому каналу. Физическое кодирование может менять форму, ширину полосы частот и гармонический состав сигнала в целях осуществления синхронизации приёмника и передатчика, устранения постоянной составляющей или уменьшения аппаратных затрат передачи сигнала.

Обнаружение ошибок в технике связи — действие, направленное на контроль целостности данных при записи/воспроизведении информации или при её передаче по линиям связи. Исправление ошибок (коррекция ошибок) — процедура восстановления информации после чтения её из устройства хранения или канала связи.

Для обнаружения ошибок используют коды обнаружения ошибок, для исправления — корректирующие коды (коды, исправляющие ошибки, коды с коррекцией ошибок, помехоустойчивые коды).

\subsection{Типы помехоустойчивого кодирования}
\subsubsection{Кодирование Хэмминга}
Коды Хемминга — простейшие линейные коды с минимальным расстоянием 3, то есть способные исправить одну ошибку. Код Хемминга может быть представлен в таком виде, что синдром
\begin{equation}
\vec{s} = \vec{r} H^T
\end{equation}

Этот принятый вектор будет равен номеру позиции, в которой произошла ошибка. Это свойство позволяет сделать декодирование очень простым.

Коды, в которых возможно автоматическое исправление ошибок, называются самокорректирующимися. Коды Хэмминга являются самоконтролирующимися кодами, то есть кодами, позволяющими автоматически обнаруживать ошибки при передаче данных и исправлять их.

Для построения самокорректирующегося кода, рассчитанного на исправление одиночных ошибок, одного контрольного разряда недостаточно. Как видно из дальнейшего, количество контрольных разрядов k должно быть выбрано так, чтобы удовлетворялось неравенство 
\begin{equation}
2^{k}\geq k+m+1
\end{equation}
или  
\begin{equation}
k \geq \log _{2}(k+m+1) 
\end{equation}
где m — количество основных двоичных разрядов кодового слова.

Минимальные значения k при заданных значениях m, найденные в соответствии с этим неравенством, приведены в таблице.
\begin{figure}[H]
    \begin{center}
        \includegraphics[width=0.2\linewidth]{table_min.png}
        \caption{Значения $K_{min}$ в зависимости от $m$} %% подпись к рисунку
        \label{table_min} %% метка рисунка для ссылки на него
    \end{center}
\end{figure}

Построение кодов Хэмминга основано на принципе проверки на четность числа единичных символов: к последовательности добавляется такой элемент, чтобы число единичных символов в получившейся последовательности было четным.

\begin{equation}
    r_1 = i_1 \oplus i_2 \oplus ... \oplus i_k
\end{equation}

\begin{equation}
    S = i_1 \oplus i_2 \oplus ... \oplus i_n \oplus r_1
\end{equation}
Тогда если $S = 0$ - ошибки нет, иначе есть однократная ошибка.

Такой код называется $(k+1,k)$. Первое число — количество элементов последовательности, второе — количество информационных символов.

Получение кодового слова выглядит следующим образом:
\begin{equation}
( i_1 \> i_2 \> i_3 \> i_4 )  \begin{pmatrix}
1 & 0 & 0 & 0 & 1 & 0 & 1 \\
0 & 1 & 0 & 0 & 1 & 1 & 1 \\         
0 & 0 & 1 & 0 & 1 & 1 & 0 \\
0 & 0 & 0 & 1 & 0 & 1 & 1
\end{pmatrix} = ( i_1 \> i_2 \> i_3 \> i_4  \> r_1  \> r_2  \> r_3)
\end{equation}

Получение синдрома выглядит следующим образом:

\begin{equation}
(i_{1} \> i_{2} \> i_{3} \> i_{4} \> r_{1} \> r_{2} \> r_{3} )  \begin{pmatrix}
1 & 0 & 1 \\
1 & 1 & 1 \\
1 & 1 & 0 \\
0 & 1 & 1 \\
1 & 0 & 0 \\
0 & 1 & 0 \\
0 & 0 & 1 \\ 
\end{pmatrix} = \begin{pmatrix}S_{1}&S_{2}&S_{3}\\\end{pmatrix}
\end{equation}

\subsubsection{Циклические коды}
Циклический код — линейный код, обладающий свойством цикличности, то есть каждая циклическая перестановка кодового слова также является кодовым словом. Используется для преобразования информации для защиты её от ошибок.

\subsubsection{Коды БЧХ}
Коды Боуза — Чоудхури — Хоквингема (БЧХ-коды) — в теории кодирования это широкий класс циклических кодов, применяемых для защиты информации от ошибок. Отличается возможностью построения кода с заранее определёнными корректирующими свойствами, а именно, минимальным кодовым расстоянием. Частным случаем БЧХ-кодов является код Рида — Соломона.

\subsubsection{Коды Рида-Соломона}
Коды Рида—Соломона (англ. Reed–Solomon codes) — недвоичные циклические коды, позволяющие исправлять ошибки в блоках данных. Элементами кодового вектора являются не биты, а группы битов (блоки).
Код Рида—Соломона является частным случаем БЧХ-кода.

\section{Ход работы}

\subsection{Коды Хэмминга}
Ниже представлены сообщение и его код(использовался стандартный код Хемминга (7,4)).
\begin{figure}[H]
    \begin{center}
        \includegraphics[width=1\linewidth]{hamm.png}
        \caption{Исходное сообщение и его код Хэмминга} %% подпись к рисунку
        \label{Ham_Code} %% метка рисунка для ссылки на него
    \end{center}
\end{figure}
При кодировании сообщений с кодовым расстоянием, равным 1, получали, как пример, закодированные сообщения с кодовым расстоянием равным 3.

\subsection{Циклические коды}
Ниже представлено сообщение, закодированное циклическим кодом, полученным стандартной функцией  Matlab (использовался стандартный код (7,4)).
\begin{figure}[H]
    \begin{center}
        \includegraphics[width=1\linewidth]{cycl.png}
        \caption{Исходное сообщение и его циклический код} %% подпись к рисунку
        \label{Cyclic_Code} %% метка рисунка для ссылки на него
    \end{center}
\end{figure}
При кодировании сообщений с кодовым расстоянием, равным 1, получали, как пример, закодированные сообщения с кодовым расстоянием равным 3.

\subsection{Коды БЧХ}
Для кодирования/декодирования с помощью кодов БЧХ используем также встроенные функции Matlab
При кодировании сообщений с кодовым расстоянием, равным 1, получали, как пример, закодированные сообщения с кодовым расстоянием равным 3, или 4.
Массивы до и после декодирования
\begin{figure}[H]
    \begin{center}
        \includegraphics[width=1\linewidth]{bch02.png}
        \caption{Массив до и после декодирования} %% подпись к рисунку
        \label{Bch_decode} %% метка рисунка для ссылки на него
    \end{center}
\end{figure}


\subsection{Коды Рида-Соломона}
При использовании кодов Рида-Соломона в виде стандартной функции rsenc можно наблюдать вектор cnumerr, который содержит количества исправляемых ошибок.

\begin{figure}[H]
    \begin{center}
        \includegraphics[width=1\linewidth]{reeds.png}
        \caption{Исходное сообщение и его циклический код} %% подпись к рисунку
        \label{Cyclic_Code} %% метка рисунка для ссылки на него
    \end{center}
\end{figure}

При кодировании сообщений с кодовым расстоянием, равным 1, получали, как пример, закодированные сообщения с кодовым расстоянием равным 3, или 4.


Реализация различных типов кодирования с помощью MATLAB:
\lstinputlisting[
    label=code:code,
    caption={Программа в МатЛаб},% для печати символ '_' требует выходной символ '\'
]{lab07.m}


\section{Выводы}
Кодирование - важный процесс при передаче сигналов по каналам связи. Методы кодирования дополняют методы модуляции для обеспечения улучшения качества передачи, для предотвращения ошибок при передаче, а также защищенности данных от получения злоумышлинниками.
В результате проделланой работы нам удалось познакомиться и разобраться с приципами различных методов кодирования, таких как:
\begin{itemize}
	\item коды Хэмминга
	\item циклические коды
	\item коды БЧХ
	\item коды Рида-Соломона
\end{itemize}
Следует отметить, что данные коды - самокорректирующие(помехоустойчивые), что является главным отличием и практическим 
подспорием при кодировании информации и передачи с возможным возникновением ошибок.
\end{document}
